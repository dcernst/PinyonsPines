\documentclass[11pt]{scrartcl}
\usepackage[scale=1.5]{ccicons}
\usepackage[notextcomp]{kpfonts} 
\usepackage[margin=1in]{geometry}
\usepackage{amsthm,amssymb}
\usepackage{graphicx}
\usepackage{enumitem}
\usepackage{bm}
\usepackage{tabu}
\usepackage{tikz}

\usepackage{color}
\definecolor{darkblue}{rgb}{0, 0, .6}
\definecolor{grey}{rgb}{.7, .7, .7}
\usepackage[breaklinks]{hyperref}
\hypersetup{
	colorlinks=true,
	linkcolor=darkblue,
	anchorcolor=darkblue,
	citecolor=darkblue,
	pagecolor=darkblue,
	urlcolor=darkblue,
	pdftitle={},
	pdfauthor={}
}

\usepackage{fancyhdr}
\pagestyle{fancy}
\lhead{MAT 612 - Spring 2016}
\chead{}
\rhead{Due Wednesday, February 24}
\renewcommand{\headrulewidth}{.4pt}

\theoremstyle{definition}
\newtheorem{theorem}{Theorem}
\newtheorem{acknowledgement}[theorem]{Acknowledgement}
\newtheorem{algorithm}[theorem]{Algorithm}
\newtheorem{axiom}[theorem]{Axiom}
\newtheorem{case}[theorem]{Case}
\newtheorem{claim}[theorem]{Claim}
\newtheorem*{claim*}{Claim}
\newtheorem{conclusion}[theorem]{Conclusion}
\newtheorem{condition}[theorem]{Condition}
\newtheorem{conjecture}[theorem]{Conjecture}
\newtheorem{corollary}[theorem]{Corollary}
\newtheorem{criterion}[theorem]{Criterion}
\newtheorem{definition}[theorem]{Definition}
\newtheorem{example}[theorem]{Example}
\newtheorem{exercise}[theorem]{Exercise}
\newtheorem{journal}[theorem]{Journal}
\newtheorem{lemma}[theorem]{Lemma}
\newtheorem{notation}[theorem]{Notation}
\newtheorem{problem}[theorem]{Problem}
\newtheorem{proposition}[theorem]{Proposition}
\newtheorem{remark}[theorem]{Remark}
\newtheorem{solution}[theorem]{Solution}
\newtheorem{summary}[theorem]{Summary}
\newtheorem{skeleton}[theorem]{Skeleton Proof}
\newtheorem{activity}[theorem]{Activity}
\newtheorem{intuitivedef}[theorem]{Intuitive Definition}

\DeclareMathOperator{\Aut}{Aut}
\DeclareMathOperator{\Inn}{Inn}
\DeclareMathOperator{\Stab}{Stab}

\newcommand{\blankline}{\pagebreak[2]\vspace{.5\baselineskip}}

\setlength{\parindent}{0pt}

%Useful for cut and paste
%\begin{enumerate}[label=\rm{(\alph*)}]

\begin{document}

\title{Homework 5}
\subtitle{Abstract Algebra II}
\date{}

\maketitle
\thispagestyle{fancy}

Complete the following problems. Note that you should only use results that we've discussed so far this semester or last semester.

%\blankline
%
%For Problems 3--7, assume that $F$ is a field.

\begin{problem}
Prove that $\mathbb{Z}[2i]$ is not a UFD.
\end{problem}

\begin{problem}
Let $p\in \mathbb{Z}$ be prime and let $f(x)\in \mathbb{Z}[x]$.  Determine general conditions under which $(p,f(x))_{\mathbb{Z}[x]}/(p)_{\mathbb{Z}[x]}$ is isomorphic to $(f(x))_{\mathbb{Z}/p\mathbb{Z}[x]}$ and prove that your answer is correct.
\end{problem}

\begin{problem}
Identify the the following rings. That is, describe them in simpler terms.
\begin{enumerate}[label=\rm{(\alph*)}]
\item $\mathbb{Z}[x]/(2,2x-1)$
\item $\mathbb{Z}[x]/(4,2x-1)$
\end{enumerate}
\end{problem}

\begin{problem}
Consider $p(x)=x^3+9x+6\in\mathbb{Q}[x]$.
\begin{enumerate}[label=\rm{(\alph*)}]
\item Show that $p(x)$ is irreducible in $\mathbb{Q}[x]$.
\item If $\theta$ is a root of $p(x)$, find the inverse of $1+\theta$ in $\mathbb{Q}(\theta)$.
\end{enumerate}
\end{problem}

\begin{problem}
Consider $p(x)=x^3+x+1\in\mathbb{Z}/2\mathbb{Z}[x]$.
\begin{enumerate}[label=\rm{(\alph*)}]
\item Show that $p(x)$ is irreducible in $\mathbb{Z}/2\mathbb{Z}[x]$.
\item If $\theta$ is a root of $p(x)$, computer the powers of $\theta$ in $\mathbb{Z}/2\mathbb{Z}(\theta)$.
\end{enumerate}
\end{problem}

\begin{problem}
Prove that $x^5-ax-1\in\mathbb{Z}[x]$ is irreducible unless $a=0,2$, or $-1$.  The first two correspond to linear factors, the third corresponds to the factorization $(x^2-x+1)(x^3+x^2-1)$.
\end{problem}

\end{document}