\documentclass[11pt]{scrartcl}
\usepackage[scale=1.5]{ccicons}
\usepackage[notextcomp]{kpfonts} 
\usepackage[margin=1in]{geometry}
\usepackage{amsthm,amssymb}
\usepackage{graphicx}
\usepackage{enumitem}
\usepackage{bm}
\usepackage{tabu}
\usepackage{tikz}

\usepackage{color}
\definecolor{darkblue}{rgb}{0, 0, .6}
\definecolor{grey}{rgb}{.7, .7, .7}
\usepackage[breaklinks]{hyperref}
\hypersetup{
	colorlinks=true,
	linkcolor=darkblue,
	anchorcolor=darkblue,
	citecolor=darkblue,
	pagecolor=darkblue,
	urlcolor=darkblue,
	pdftitle={},
	pdfauthor={}
}

\usepackage{fancyhdr}
\pagestyle{fancy}
\lhead{MAT 612 - Spring 2016}
\chead{}
\rhead{Due Wednesday, April 27}
\renewcommand{\headrulewidth}{.4pt}

\theoremstyle{definition}
\newtheorem{theorem}{Theorem}
\newtheorem{acknowledgement}[theorem]{Acknowledgement}
\newtheorem{algorithm}[theorem]{Algorithm}
\newtheorem{axiom}[theorem]{Axiom}
\newtheorem{case}[theorem]{Case}
\newtheorem{claim}[theorem]{Claim}
\newtheorem*{claim*}{Claim}
\newtheorem{conclusion}[theorem]{Conclusion}
\newtheorem{condition}[theorem]{Condition}
\newtheorem{conjecture}[theorem]{Conjecture}
\newtheorem{corollary}[theorem]{Corollary}
\newtheorem{criterion}[theorem]{Criterion}
\newtheorem{definition}[theorem]{Definition}
\newtheorem{example}[theorem]{Example}
\newtheorem{exercise}[theorem]{Exercise}
\newtheorem{journal}[theorem]{Journal}
\newtheorem{lemma}[theorem]{Lemma}
\newtheorem{notation}[theorem]{Notation}
\newtheorem{problem}[theorem]{Problem}
\newtheorem{proposition}[theorem]{Proposition}
\newtheorem{remark}[theorem]{Remark}
\newtheorem{solution}[theorem]{Solution}
\newtheorem{summary}[theorem]{Summary}
\newtheorem{skeleton}[theorem]{Skeleton Proof}
\newtheorem{activity}[theorem]{Activity}
\newtheorem{intuitivedef}[theorem]{Intuitive Definition}

\DeclareMathOperator{\Aut}{Aut}
\DeclareMathOperator{\Gal}{Gal}
\DeclareMathOperator{\Inn}{Inn}
\DeclareMathOperator{\Stab}{Stab}
\DeclareMathOperator{\Char}{Char}

\newcommand{\blankline}{\pagebreak[2]\vspace{.5\baselineskip}}

\setlength{\parindent}{0pt}

%Useful for cut and paste
%\begin{enumerate}[label=\rm{(\alph*)}]

\begin{document}

\title{Homework 10}
\subtitle{Abstract Algebra II}
\date{}

\maketitle
\thispagestyle{fancy}

Complete the following problems. Note that you should only use results that we've discussed so far this semester or last semester.

\begin{problem}
Determine all the subfields of the splitting field of $x^8-2$ that are Galois over $\mathbb{Q}$.  \emph{Note:} You are welcome to consult the example involving the splitting field of $x^8-2$, which appears at the end of Section 14.2 of Dummit and Foote.
\end{problem}

%\begin{problem}
%Determine the order of Galois group of the splitting field over $\mathbb{Q}$ of $x^8-3$.
%\end{problem}

\begin{problem}
Suppose $K/F$ is Galois such that $[K:F]=p^n$ for some prime $p$ and $n\geq 1$. Prove that there are Galois extensions of $F$ contained in $K$ of degrees $p$ and $p^{n-1}$.
\end{problem}

\begin{problem}
Give an example of fields $F_1, F_2, F_3$ with $\mathbb{Q}\subset F_1\subset F_2\subset F_3$, $[F_3:\mathbb{Q}]=8$, and each field is Galois over all its subfields with the exception of that $F_2$ is not Galois over $\mathbb{Q}$.
\end{problem}

\begin{problem}
Consider the extension $\mathbb{Q}(\sqrt{2+\sqrt{2}})/\mathbb{Q}$.
\begin{enumerate}[label=\rm{(\alph*)}]
\item Prove that $\mathbb{Q}(\sqrt{2+\sqrt{2}})/\mathbb{Q}$ is a Galois extension of degree 4
\item Exhibit the Galois correspondence of the subfields of $\mathbb{Q}(\sqrt{2+\sqrt{2}})$ containing $\mathbb{Q}$ with the subgroups of the Galois group of $\mathbb{Q}(\sqrt{2+\sqrt{2}})/\mathbb{Q}$.
\item Determine which subfields of $\mathbb{Q}(\sqrt{2+\sqrt{2}})$ are Galois over $\mathbb{Q}$.
\end{enumerate}
\end{problem}

\begin{problem}
Consider the separable polynomial $f(x)=x^4-12x^2+35$ over $\mathbb{Q}$
\begin{enumerate}[label=\rm{(\alph*)}]
\item Determine the Galois group over $\mathbb{Q}$ of $f(x)$.
\item Exhibit the Galois correspondence of the subfields of the splitting field of $f(x)$ containing $\mathbb{Q}$ with the subgroups of the Galois group of $f(x)$.
\item Determine which subfields of the splitting field of $f(x)$ are Galois over $\mathbb{Q}$.
\end{enumerate} 
\end{problem}

\begin{problem}
Consider the separable polynomial $g(x)=x^4-2$ over $\mathbb{Q}$
\begin{enumerate}[label=\rm{(\alph*)}]
\item Determine the Galois group over $\mathbb{Q}$ of $g(x)$.
\item Exhibit the Galois correspondence of the subfields of the splitting field of $g(x)$ containing $\mathbb{Q}$ with the subgroups of the Galois group of $g(x)$.
\item Determine which subfields of the splitting field of $g(x)$ are Galois over $\mathbb{Q}$.
\end{enumerate} 
\end{problem}

%\begin{problem}
%Prove that $\mathbb{Q}(\sqrt[8]{2},i)/\mathbb{Q}(\sqrt{-2})$ is Galois with Galois group isomorphic to $Q_8$.
%\end{problem}

\end{document}