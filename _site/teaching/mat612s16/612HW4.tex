\documentclass[11pt]{scrartcl}
\usepackage[scale=1.5]{ccicons}
\usepackage[notextcomp]{kpfonts} 
\usepackage[margin=1in]{geometry}
\usepackage{amsthm,amssymb}
\usepackage{graphicx}
\usepackage{enumitem}
\usepackage{bm}
\usepackage{tabu}
\usepackage{tikz}

\usepackage{color}
\definecolor{darkblue}{rgb}{0, 0, .6}
\definecolor{grey}{rgb}{.7, .7, .7}
\usepackage[breaklinks]{hyperref}
\hypersetup{
	colorlinks=true,
	linkcolor=darkblue,
	anchorcolor=darkblue,
	citecolor=darkblue,
	pagecolor=darkblue,
	urlcolor=darkblue,
	pdftitle={},
	pdfauthor={}
}

\usepackage{fancyhdr}
\pagestyle{fancy}
\lhead{MAT 612 - Spring 2016}
\chead{}
\rhead{Due Wednesday, February 17}
\renewcommand{\headrulewidth}{.4pt}

\theoremstyle{definition}
\newtheorem{theorem}{Theorem}
\newtheorem{acknowledgement}[theorem]{Acknowledgement}
\newtheorem{algorithm}[theorem]{Algorithm}
\newtheorem{axiom}[theorem]{Axiom}
\newtheorem{case}[theorem]{Case}
\newtheorem{claim}[theorem]{Claim}
\newtheorem*{claim*}{Claim}
\newtheorem{conclusion}[theorem]{Conclusion}
\newtheorem{condition}[theorem]{Condition}
\newtheorem{conjecture}[theorem]{Conjecture}
\newtheorem{corollary}[theorem]{Corollary}
\newtheorem{criterion}[theorem]{Criterion}
\newtheorem{definition}[theorem]{Definition}
\newtheorem{example}[theorem]{Example}
\newtheorem{exercise}[theorem]{Exercise}
\newtheorem{journal}[theorem]{Journal}
\newtheorem{lemma}[theorem]{Lemma}
\newtheorem{notation}[theorem]{Notation}
\newtheorem{problem}[theorem]{Problem}
\newtheorem{proposition}[theorem]{Proposition}
\newtheorem{remark}[theorem]{Remark}
\newtheorem{solution}[theorem]{Solution}
\newtheorem{summary}[theorem]{Summary}
\newtheorem{skeleton}[theorem]{Skeleton Proof}
\newtheorem{activity}[theorem]{Activity}
\newtheorem{intuitivedef}[theorem]{Intuitive Definition}

\DeclareMathOperator{\Aut}{Aut}
\DeclareMathOperator{\Inn}{Inn}
\DeclareMathOperator{\Stab}{Stab}

\newcommand{\blankline}{\pagebreak[2]\vspace{.5\baselineskip}}

\setlength{\parindent}{0pt}

%Useful for cut and paste
%\begin{enumerate}[label=\rm{(\alph*)}]

\begin{document}

\title{Homework 4}
\subtitle{Abstract Algebra II}
\date{}

\maketitle
\thispagestyle{fancy}

Complete the following problems. Note that you should only use results that we've discussed so far this semester or last semester.

%\blankline
%
%For Problems 3--7, assume that $F$ is a field.

\begin{problem}
Determine all ideals of the ring $\mathbb{Z}[x]/(2,x^3+1)$.
\end{problem}

\begin{problem}
Prove that if $f(x),g(x)\in\mathbb{Q}[x]$ such that $f(x)g(x)\in\mathbb{Z}[x]$, then the product of any coefficient of $f(x)$ with any coefficient of $g(x)$ is an integer.
\end{problem}

%\begin{problem}
%Prove that $\mathbb{Z}[2i]$ is not a UFD.
%\end{problem}

\begin{problem}
Determine whether each of the following polynomials is irreducible in the given ring. Justify your answers. If a polynomial is reducible, write it as a product of irreducibles.
\begin{enumerate}[label=\rm{(\alph*)}]
\item $x^4+1$ in $\mathbb{Z}/5\mathbb{Z}[x]$.
\item $x^4+10x^2+1$ in $\mathbb{Z}[x]$.
\item $x^4-4x^3+6$ in $\mathbb{Z}[x]$.
\item $x^6+30x^5-15x^3+6x-120$ in $\mathbb{Z}[x]$.
\end{enumerate}
\end{problem}

\begin{problem}
Prove that the polynomial $(x-1)(x-2)\cdots (x-n)-1$ is irreducible over $\mathbb{Z}$ for all $n\geq 1$.
\end{problem}

\begin{problem}
Prove that the ring $\mathbb{R}[x]/(x^2+1)$ is isomorphic to the field $\mathbb{C}$.
\end{problem}

\begin{problem}
Prove that the polynomial $x^2-\sqrt{2}$ is irreducible over $\mathbb{Z}[\sqrt{2}]$. \emph{Note:} You may assume that $\mathbb{Z}[\sqrt{2}]$ is a UFD.
\end{problem}

\begin{problem}
Prove that $x^2+y^2-1$ is irreducible over $\mathbb{Q}[x,y]$. 
\end{problem}

\end{document}