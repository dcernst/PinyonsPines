\documentclass[11pt]{scrartcl}
\usepackage[scale=1.5]{ccicons}
\usepackage[notextcomp]{kpfonts} 
\usepackage[margin=1in]{geometry}
\usepackage{amsthm,amssymb}
\usepackage{graphicx}
\usepackage{enumitem}
\usepackage{bm}
\usepackage{tabu}
\usepackage{tikz}

\usepackage{color}
\definecolor{darkblue}{rgb}{0, 0, .6}
\definecolor{grey}{rgb}{.7, .7, .7}
\usepackage[breaklinks]{hyperref}
\hypersetup{
	colorlinks=true,
	linkcolor=darkblue,
	anchorcolor=darkblue,
	citecolor=darkblue,
	pagecolor=darkblue,
	urlcolor=darkblue,
	pdftitle={},
	pdfauthor={}
}

\usepackage{fancyhdr}
\pagestyle{fancy}
\lhead{MAT 511 - Fall 2015}
\chead{}
\rhead{Due Friday, November 13}
\renewcommand{\headrulewidth}{.4pt}

\theoremstyle{definition}
\newtheorem{theorem}{Theorem}
\newtheorem{acknowledgement}[theorem]{Acknowledgement}
\newtheorem{algorithm}[theorem]{Algorithm}
\newtheorem{axiom}[theorem]{Axiom}
\newtheorem{case}[theorem]{Case}
\newtheorem{claim}[theorem]{Claim}
\newtheorem*{claim*}{Claim}
\newtheorem{conclusion}[theorem]{Conclusion}
\newtheorem{condition}[theorem]{Condition}
\newtheorem{conjecture}[theorem]{Conjecture}
\newtheorem{corollary}[theorem]{Corollary}
\newtheorem{criterion}[theorem]{Criterion}
\newtheorem{definition}[theorem]{Definition}
\newtheorem{example}[theorem]{Example}
\newtheorem{exercise}[theorem]{Exercise}
\newtheorem{journal}[theorem]{Journal}
\newtheorem{lemma}[theorem]{Lemma}
\newtheorem{notation}[theorem]{Notation}
\newtheorem{problem}[theorem]{Problem}
\newtheorem{proposition}[theorem]{Proposition}
\newtheorem{remark}[theorem]{Remark}
\newtheorem{solution}[theorem]{Solution}
\newtheorem{summary}[theorem]{Summary}
\newtheorem{skeleton}[theorem]{Skeleton Proof}
\newtheorem{activity}[theorem]{Activity}
\newtheorem{intuitivedef}[theorem]{Intuitive Definition}

\DeclareMathOperator{\Aut}{Aut}
\DeclareMathOperator{\Stab}{Stab}

\newcommand{\blankline}{\pagebreak[2]\vspace{.5\baselineskip}}

\setlength{\parindent}{0pt}

%Useful for cut and paste
%\begin{enumerate}[label=\rm{(\alph*)}]

\begin{document}

\title{Homework 9}
\subtitle{Abstract Algebra I}
\date{}

\maketitle
\thispagestyle{fancy}

Complete the following problems. Note that you should only use results that we've discussed so far this semester.

\begin{problem}
Find all conjugacy classes and their sizes for the following groups.
\begin{enumerate}[label=\rm{(\alph*)}]
\item $D_8$
\item $Q_8$
\item $A_4$
\end{enumerate}
\end{problem}

\begin{problem}
If $[G:Z(G)]=n$, prove that every conjugacy class has at most $n$ elements.
\end{problem}

\begin{problem}
Assume $G$ is a non-abelian group of order 15.
\begin{enumerate}[label=\rm{(\alph*)}]
\item Prove that the center of $G$ is trivial.
\item Use the fact that $\langle g\rangle \leq C_G(g)$ for all $g\in G$ to show that there is at most one possible class equation for $G$. 
\end{enumerate}
\emph{Hint:} Use my favorite problem.
\end{problem}

\begin{problem}
Prove that the center of $S_n$ is trivial for all $n\geq 3$.
\end{problem}

\begin{problem}
Find all finite groups that have exactly two conjugacy clases.
\end{problem}

\begin{problem}
Prove that if $n$ is odd, then the set of all $n$-cycles consists of two conjugacy classes of equal size in $A_n$.
\end{problem}

\begin{problem}
A proper subgroup $M$ of a group $G$ is called \emph{maximal} if whenever $M\leq H\leq G$, either $H=M$ or $H=G$. 
\begin{enumerate}[label=\rm{(\alph*)}]
\item Prove that if $M$ is a maximal subgroup of $G$, then either $N_G(M)=M$ or $N_G(M)=G$.
\item Prove that if $M$ is a maximal subgroup of $G$ that is not normal in $G$, then the number of nonidentity elements of $G$ that are contained in conjugates of $M$ is at most $(|M|-1)[G:M]$.
\end{enumerate}
\end{problem}

\begin{problem}
Let $H$ be a proper subgroup of a finite group $G$. Prove that $G\neq \cup_{g\in G}gHg^{-1}$. \emph{Hint:} Use the previous problem.
\end{problem}

\begin{problem}
Let $g_1,g_2,\ldots, g_r$ be representatives of the conjugacy classes of the finite group $G$ and assume these elements pairwise commute. Prove that $G$ is abelian.
\end{problem}

\begin{problem}
Let $p$ be a prime and let $G$ be a group of order $p^{\alpha}$. Prove that $G$ has a subgroup of order $p^{\beta}$, for every $\beta$ with $0\leq \beta\leq \alpha$.
\end{problem}

\end{document}