\documentclass[11pt]{scrartcl}
\usepackage[scale=1.5]{ccicons}
\usepackage[notextcomp]{kpfonts} 
\usepackage[margin=1in]{geometry}
\usepackage{amsthm,amssymb}
\usepackage{graphicx}
\usepackage{enumitem}
\usepackage{bm}
\usepackage{tabu}
\usepackage{tikz}

\usepackage{color}
\definecolor{darkblue}{rgb}{0, 0, .6}
\definecolor{grey}{rgb}{.7, .7, .7}
\usepackage[breaklinks]{hyperref}
\hypersetup{
	colorlinks=true,
	linkcolor=darkblue,
	anchorcolor=darkblue,
	citecolor=darkblue,
	pagecolor=darkblue,
	urlcolor=darkblue,
	pdftitle={},
	pdfauthor={}
}

\usepackage{fancyhdr}
\pagestyle{fancy}
\lhead{MAT 511 - Fall 2015}
\chead{}
\rhead{Due Friday, October 9}
%\lfoot{}%\scriptsize This work is licensed under the \href{http://creativecommons.org/licenses/by-sa/3.0/us/}{Creative Commons Attribution-Share Alike 3.0 License}.} 
%\cfoot{}
%\rfoot{\ccbysa}
\renewcommand{\headrulewidth}{.4pt}
%\renewcommand{\footrulewidth}{.4pt}

\theoremstyle{definition}
\newtheorem{theorem}{Theorem}
\newtheorem{acknowledgement}[theorem]{Acknowledgement}
\newtheorem{algorithm}[theorem]{Algorithm}
\newtheorem{axiom}[theorem]{Axiom}
\newtheorem{case}[theorem]{Case}
\newtheorem{claim}[theorem]{Claim}
\newtheorem*{claim*}{Claim}
\newtheorem{conclusion}[theorem]{Conclusion}
\newtheorem{condition}[theorem]{Condition}
\newtheorem{conjecture}[theorem]{Conjecture}
\newtheorem{corollary}[theorem]{Corollary}
\newtheorem{criterion}[theorem]{Criterion}
\newtheorem{definition}[theorem]{Definition}
\newtheorem{example}[theorem]{Example}
\newtheorem{exercise}[theorem]{Exercise}
\newtheorem{journal}[theorem]{Journal}
\newtheorem{lemma}[theorem]{Lemma}
\newtheorem{notation}[theorem]{Notation}
\newtheorem{problem}[theorem]{Problem}
\newtheorem{proposition}[theorem]{Proposition}
\newtheorem{remark}[theorem]{Remark}
\newtheorem{solution}[theorem]{Solution}
\newtheorem{summary}[theorem]{Summary}
\newtheorem{skeleton}[theorem]{Skeleton Proof}
\newtheorem{activity}[theorem]{Activity}
\newtheorem{intuitivedef}[theorem]{Intuitive Definition}

\DeclareMathOperator{\Aut}{Aut}

\newcommand{\blankline}{\pagebreak[2]\vspace{.5\baselineskip}}

\setlength{\parindent}{0pt}

%Useful for cut and paste
%\begin{enumerate}[label=\rm{(\alph*)}]

\begin{document}

\title{Homework 6}
\subtitle{Abstract Algebra I}
\date{}

\maketitle
\thispagestyle{fancy}

Complete the following problems. Note that you should only use results that we've discussed so far this semester. 

\begin{problem}
Determine whether each of the specified subsets is a subgroup of the given group. If the subset is a subgroup, prove it.  If the subset is not a subgroup, explain why.
\begin{enumerate}[label=\rm{(\alph*)}]
\item The set of reflections from $D_{2n}$.
\item $\{a+ai\mid a\in\mathbb{R}\}\subseteq\mathbb{C}$ (under addition).
\item $\{z\in\mathbb{C}\mid |z|=1\}\subseteq\mathbb{C}\setminus\{0\}$ (under multiplication).
\item $\{x\in\mathbb{R}\mid x^2\in\mathbb{Q}\}\subseteq\mathbb{R}$ (under addition).
\end{enumerate}
\end{problem}

\begin{problem}
Suppose $H$ and $K$ are subgroups of $G$.  Prove that $H\cap K$ is also a subgroup of $G$.
\end{problem}

\begin{problem}
Given an example of an infinite group $G$ and an infinite subset $H$ of $G$ such that $H$ is closed under the operation of $G$ but is not a subgroup of $G$.
\end{problem}

\begin{problem}
Let $G$ be an abelian group.  
\begin{enumerate}[label=\rm{(\alph*)}]
\item Prove that $\{g\in G\mid |g|<\infty\}$ is a subgroup of $G$ (called the \emph{torsion subgroup of $G$}).  
\item Give an example of a group $G$ where the set described above is not a subgroup. Briefly justify your answer.  \emph{Hint:}  Fiddle around some infinite non-abelian groups.
\end{enumerate} 
\end{problem}

\begin{problem}
For each of the following groups, compute the centralizers of each element and find the center of each group.
\begin{enumerate}[label=\rm{(\alph*)}]
\item $S_3$
\item $D_8$
\item $Q_8$
\end{enumerate}
\end{problem}

\begin{problem}
Compute the normalizer for each subgroup of $D_8$.  \emph{Note:} There are 10 subgroups of $D_8$.
\end{problem}

\begin{problem}
Let $H\leq G$.  Prove one of the following.
\begin{enumerate}[label=\rm{(\alph*)}]
\item Prove that $H\leq N_G(H)$.
\item Prove that $H\leq C_G(H)$ iff $H$ is abelian.
\end{enumerate}
\end{problem}

\begin{problem}
Determine whether each group is cyclic.  Justify your answer.
\begin{enumerate}[label=\rm{(\alph*)}]
\item $\mathbb{Z}\times \mathbb{Z}$.
\item $\mathbb{Q}$
\end{enumerate}
\end{problem}

\begin{problem}
Provide an example of a group $G$ such that every proper subgroup of $G$ is cyclic, but $G$ is not cyclic.
\end{problem}

\begin{problem}
Give an example of two sets $A$ and $B$ contained in a group $G$ such that (i) $A\subseteq B$, (ii) $A\neq B$, and (iii) $\langle A\rangle=\langle B\rangle$.
\end{problem}

\begin{problem}
Let $C_n$ be a cyclic group of order $n$. Fix $a\in\mathbb{Z}$.  Define $\sigma_a:C_n\to C_n$ via $\sigma_a(x)=x^a$.  Prove that $\sigma_a$ is an automorphism of $C_n$ iff $a$ and $n$ are relatively prime.
\end{problem}

\begin{problem}
Prove one of the following.
\begin{enumerate}[label=\rm{(\alph*)}]
\item Prove that the subgroup $\langle(1,2), (1,3)(2,4)\rangle$ of $S_4$ is isomorphic to $D_8$.
\item Prove that the subgroup $\langle s,r^2\rangle$ of $D_8$ is isomorphic to $\mathbb{Z}_2\times \mathbb{Z}_2$.
\end{enumerate}
\end{problem}

\begin{problem}
Consider the group $\mathbb{Z}/48\mathbb{Z}$.
\begin{enumerate}[label=\rm{(\alph*)}]
\item Find all generators for $\mathbb{Z}/48\mathbb{Z}$.
\item What is the order of $\overline{30}$ in $\mathbb{Z}/48\mathbb{Z}$?
\item Draw the subgroup lattice for $\mathbb{Z}/48\mathbb{Z}$.
\end{enumerate}
\end{problem}
\end{document}