\documentclass[11pt]{scrartcl}
\usepackage[scale=1.5]{ccicons}
\usepackage[notextcomp]{kpfonts} 
\usepackage[margin=1in]{geometry}
\usepackage{amsthm,amssymb}
\usepackage{graphicx}
\usepackage{enumitem}
\usepackage{bm}
\usepackage{tabu}
\usepackage{tikz}

\usepackage{color}
\definecolor{darkblue}{rgb}{0, 0, .6}
\definecolor{grey}{rgb}{.7, .7, .7}
\usepackage[breaklinks]{hyperref}
\hypersetup{
	colorlinks=true,
	linkcolor=darkblue,
	anchorcolor=darkblue,
	citecolor=darkblue,
	pagecolor=darkblue,
	urlcolor=darkblue,
	pdftitle={},
	pdfauthor={}
}

\usepackage{fancyhdr}
\pagestyle{fancy}
\lhead{MAT 511 - Fall 2015}
\chead{}
\rhead{Due Wednesday, November 18}
\renewcommand{\headrulewidth}{.4pt}

\theoremstyle{definition}
\newtheorem{theorem}{Theorem}
\newtheorem{acknowledgement}[theorem]{Acknowledgement}
\newtheorem{algorithm}[theorem]{Algorithm}
\newtheorem{axiom}[theorem]{Axiom}
\newtheorem{case}[theorem]{Case}
\newtheorem{claim}[theorem]{Claim}
\newtheorem*{claim*}{Claim}
\newtheorem{conclusion}[theorem]{Conclusion}
\newtheorem{condition}[theorem]{Condition}
\newtheorem{conjecture}[theorem]{Conjecture}
\newtheorem{corollary}[theorem]{Corollary}
\newtheorem{criterion}[theorem]{Criterion}
\newtheorem{definition}[theorem]{Definition}
\newtheorem{example}[theorem]{Example}
\newtheorem{exercise}[theorem]{Exercise}
\newtheorem{journal}[theorem]{Journal}
\newtheorem{lemma}[theorem]{Lemma}
\newtheorem{notation}[theorem]{Notation}
\newtheorem{problem}[theorem]{Problem}
\newtheorem{proposition}[theorem]{Proposition}
\newtheorem{remark}[theorem]{Remark}
\newtheorem{solution}[theorem]{Solution}
\newtheorem{summary}[theorem]{Summary}
\newtheorem{skeleton}[theorem]{Skeleton Proof}
\newtheorem{activity}[theorem]{Activity}
\newtheorem{intuitivedef}[theorem]{Intuitive Definition}

\DeclareMathOperator{\Aut}{Aut}
\DeclareMathOperator{\Inn}{Inn}
\DeclareMathOperator{\Stab}{Stab}

\newcommand{\blankline}{\pagebreak[2]\vspace{.5\baselineskip}}

\setlength{\parindent}{0pt}

%Useful for cut and paste
%\begin{enumerate}[label=\rm{(\alph*)}]

\begin{document}

\title{Homework 10}
\subtitle{Abstract Algebra I}
\date{}

\maketitle
\thispagestyle{fancy}

Complete the following problems. Note that you should only use results that we've discussed so far this semester.

%\begin{problem}
%Let $G$ be a group and consider the group $\Aut(G)$.
%\begin{enumerate}[label=\rm{(\alph*)}]
%\item If $\sigma\in \Aut(G)$ and $\phi_g$ is conjugation by $g\in G$, then prove that $\sigma\phi_g\sigma^{-1}=\phi_{\sigma(g)}$.
%\item Prove that $\Inn(G)\unlhd\Aut(G)$.
%\end{enumerate}
%\end{problem}

\begin{problem}
Prove that if $G$ is an abelian group of order $pq$, where $p$ and $q$ are distinct primes, then $G$ is cyclic. \emph{Hint:} Use Problem 1 from Homework 8.
\end{problem}

\begin{problem}
Prove that $|\Aut(D_8)|\leq 8$ by first proving that under any automorphism, $r$ has at most two possible images and $s$ has at most four possible images. 
\end{problem}

\begin{problem}
Prove that characteristic subgroups are normal.
\end{problem}

\begin{problem}
Provide an example of a normal subgroup that is not characteristic.
\end{problem}

\begin{problem}
Prove that if $H$ is the unique subgroup of a given order in $G$, then $H$ is characteristic. \emph{Hint:} This problem is as easy as quoting a previous homework problem.
\end{problem}

\begin{problem}
Prove \textbf{one} of the following.
\begin{enumerate}[label=\rm{(\alph*)}]
\item If $H$ is characteristic in $K$ and $K$ is normal in $G$, then $H$ is normal in $G$.
\item If $H$ is characteristic in $K$ and $K$ is characteristic in $G$, then $H$ is characteristic in $G$.
\end{enumerate}
\end{problem}

\end{document}