\documentclass{beamer}
%\documentclass[handout]{beamer}  USE THIS LINE IF YOU WANT TO PRINT A HANDOUT VERSION

\usepackage{amsfonts}
\usepackage{amsmath}
\usepackage{latexsym}
\usepackage{mathrsfs}
\usepackage{amssymb}
\usepackage{graphicx}

\mode<presentation>

\usetheme{shadow}
%\usecolortheme{seahorse}

\begin{document}

\title[10 Things I Wish I Would Have Known]{10 Things I Wish I Would Have Known Before I Started Teaching}
\author{Dana Ernst}
\institute{University of Colorado, Boulder\\
Department of Mathematics}
\date{November 6, 2006}
\frame{\titlepage}

\begin{frame}
\frametitle{Disclaimer:}
\begin{itemize}
	\pause 
  \item What follows is a selection of things that I've learned during the past few years of teaching
    	\pause 
  \item My hope is that you can benefit from some of the mistakes that I've made
    	\pause 
  \item Feel free to disagree
  \end{itemize}
\end{frame}


\begin{frame}
\frametitle{1. The syllabus is a contract between you and your students}
\begin{itemize}
	\pause 
  \item Your syllabus is a legal document that outlines the rules and expectations of the course
  	\pause 
  \item If you have a particular expectation, put it in the syllabus!!!
  	\pause 
  \item \alert{Never} deviate from the expectations that you laid out in the syllabus
  	\pause
  \item The only thing worse than a "bad rule" on the syllabus, is changing it mid-semester
\end{itemize}
\end{frame}

\begin{frame}
\frametitle{2. It's better to start off as the bad cop than the good cop }
\begin{itemize}
	\pause 
  \item If you start off being strict, you can always loosen up later
  	\pause 
  \item But if you start off loose, it is very difficult to increase control!
    	\pause 
  \item If given the opportunity, students will try to take advantage of you (though, this is often not intended to be malicious)
\end{itemize}
\end{frame}

\begin{frame}
\frametitle{3. Teaching is not a popularity contest}
\begin{itemize}
	\pause 
  \item New teachers are often overly concerned with whether their students like them or not
  	\pause (I know I was)
	\pause
  \item If you are too concerned with whether students like you or not, you may miss out on an opportunity to provide the students with a worthwhile and challenging learning experience
  	\pause
  \item If you are passionate about what you are teaching \pause and your students \alert{trust} you, \pause I bet that they will like you
  	\pause
  \item If you're having fun, I bet that they will like you
  \end{itemize}
\end{frame}

\begin{frame}
\frametitle{4. Students want to know "Why?"}
\begin{itemize}
	\pause 
  \item There is one question that all young kids ask over and over again: \pause \alert{Why?}
  	\pause
  \item College students haven't really lost this thirst for knowledge...\pause it's just hidden behind a little laziness and ambivalence
  	\pause 
  \item Don't just list a series of facts when teaching
  	\pause
  \item Build answers to the question "Why?" into your lectures and be prepared to answer these types of questions at all times
      	\pause 
  \item Encourage students to explore "why" on their own as often as possible
\end{itemize}
\end{frame}

\begin{frame}
\frametitle{5. It's OK to make mistakes}
\begin{itemize}
	\pause 
  \item Everyone makes mistakes, but do your best to minimize them
    	\pause 
  \item Always admit your mistakes.    \pause  Students will see through your attempt to cover it up... \pause and this may erode their trust in you
    	\pause
  \item Use your mistake as a teachable moment.  \pause Explain to them \alert{WHY} what you said is wrong
  	\pause
  \item Also, it's OK to admit that you don't know something...\pause, but you should tell them that you will find out and let them know later (make sure you follow through)
\end{itemize}
\end{frame}

\begin{frame}
\frametitle{6. A few seconds of silence seems like an eternity}
\begin{itemize}
	\pause
  \item Regardless of what class you are teaching or how you are teaching it, you should pause occasionally to let students think and/or write
    	\pause 
  \item In particular, if you ask a question to the class, wait a while before answering it for them
      	\pause 
  \item If you begin answering your own questions too soon, your students may get in the habit of not even trying to think of an answer
  	\pause
  \item Learn to feel comfortable in the silence
\end{itemize}
\end{frame}

\begin{frame}
\frametitle{7. If it's important, then write it down}
\begin{itemize}
	\pause 
  \item Students will not remember most of what you say (even if you say it a 1000 times)
    	\pause 
  \item \alert{ALL} of the important stuff should be written down (either on the chalkboard, in PowerPoint, etc).  At the very least write down where what you said can be found in writing (textbook, internet, etc.)
    	\pause 
  \item However, if students are taking notes, there is a limit to how much they can write down and still follow what's going on in class
\end{itemize}
\end{frame}

\begin{frame}
\frametitle{8. Be prepared to answer those difficult questions}
\begin{itemize}
	\pause 
  \item Sure, you should be able answer all the hard questions that might come up on the material you are teaching, but that's not what I meant.  How about these?
    	\pause 
  \item "Why do we have to learn this stuff?"
      	\pause 
  \item "Do we have to know this for the test?"
  	\pause
  \item "I don't get it!?"
\end{itemize}
\end{frame}

\begin{frame}
\frametitle{9. There are a few things that you probably shouldn't do}
\begin{itemize}
	\pause 
  \item Don't say: \pause "This is easy." \pause Of course there are exceptions to this, but I think it's best to minimize this phrase
    	\pause 
  \item Never say: \pause "I'm only teaching this because it's on the syllabus"
  	\pause
  \item Never act like you are not enjoying what you are doing (if you have to, fake it)
      	\pause 
  \item Don't yawn!
  	\pause
  \item Don't roll your eyes
  	\pause
  \item Don't look at your watch.  \pause If you look at your watch, then your students will, too
  	\pause
  \item It's probably best not to swear either... \pause I am so guilty of this!
  	\pause
 \item It's OK to emphasize the material that the students should know for exams, but don't lead them into thinking that doing well on the exam is the point
	\pause
  \item Hmmm, what is the point?
\end{itemize}
\end{frame}

\begin{frame}
\frametitle{10. Spend some time thinking about what the point is}
\begin{itemize}
	\pause 
  \item What's the point of learning all this stuff?
  	\pause
  \item What's the point of your lecture?
      	\pause 
  \item What's the purpose of the class you are teaching?
  	\pause 
  \item What's the purpose of learning anything?
  	\pause
  \item Have you told the students any of the answers to these questions?
  	\pause
  \item Or, have you asked them to think about these things?
  	\pause
  \item Do the answers to these questions change how you teach? \pause I think that they should!
\end{itemize}
\end{frame}

\begin{frame}
\frametitle{Bonus: Erase the board correctly}
\begin{itemize}
	\pause 
  \item Erase the board completely
  	\pause 
  \item Always erase the board up and down
      	\pause 
  \item Never erase the board side to side
  	\pause
  \item Otherwise, your butt will wiggle!!!
\end{itemize}
\end{frame}

\begin{frame}
\frametitle{Alright, your turn!}
\pause
Get into groups of 3 and brainstorm at least 3 more things that would be useful for all of us to know.  We'll share these ideas with each other in a few minutes.
\end{frame}

\end{document}