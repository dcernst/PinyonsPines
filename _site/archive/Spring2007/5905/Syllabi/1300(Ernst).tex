%August 2006
%
%%%%%%%%%%%%%%%%%%%%%%%%%%%%%%%%%%%
%
\documentclass[12pt]{amsart}
\usepackage{amsmath,amsthm,amssymb}
\usepackage{url}
\pagestyle{headings}
%
\setlength{\textwidth}{6.5truein}
\setlength{\oddsidemargin}{0 truept}
\setlength{\evensidemargin}{0 truept}
\setlength{\textheight}{9truein}
%

\begin{document}


\centerline{\large\bf MATH 1300-014: Calculus 1, Fall 2006} 
\centerline{\bf MTWRF 12:00--12:50 pm, MUEN D144}

%\bigskip

\bigskip

\noindent {\bf General Information:}\\

%\noindent
\begin{tabular}{ll}
{\bf Instructor:} Dana Ernst & {\bf Office:} Math 214\\
{\bf Email:} Dana.Ernst@colorado.edu & {\bf Office hours:} MWF time TBA (or by appnt)\\
{}\\
{\bf Recitation TA:} Rachel Krieger & {\bf Office:} Mathematics 242\\
{\bf Email:} Rachael.Krieger@colorado.edu & {\bf Office hours:} M 12:30-1:30, W 3-4, R 1-2\\
\end{tabular}

\medskip

\begin{tabular}{ll}
{\bf Section Webpage:} & \url{http://math.colorado.edu/~ ernstd}\\
{\bf Course Webpage:} & \url{http://spot.colorado.edu/~szendrei/Calc_F06/calc1.html}
\end{tabular}

\medskip

\noindent
{\bf Prerequisites:} 
Two years of high school algebra, 
one year of geometry, and 
one half year of trigonometry, 
{\it or} Math 1150.

\medskip

\noindent
{\bf Text:}
Anton, Bivens, \& Davis: 
{\it Single Variable Calculus, Math 1300/2300}
(Wiley)

\medskip

\noindent
{\bf Supplemental material:}
Web appendices D--H to the text, 
the Student Solutions Manual, the Student Study Guide,
and many other resources for learning are available 
through WileyPLUS (see below).

\medskip

\noindent
{\bf WileyPLUS}
is an online system that integrates various tools for learning, 
for example, reading materials 
(including the solutions manual and study guide), 
practice problems, and `Just ask' tutorials.
To access WileyPLUS, go to the course webpage,
and click on `WileyPLUS login' at the top of the left panel.


\medskip

\noindent
{\bf Homework:}
Homework assignments will be posted through
WileyPLUS, and all homeworks will have to be
submitted online through WileyPLUS by the given deadline.
No other form of submission will be accepted, and no
late homework will be accepted.
The six lowest homework scores will not count
towards the final grade.  As a rule, homework will be assigned Monday, Wednesday, Friday,
and will be due at 7 a.m. before the second succeeding
MWF lecture.
The homework assignments will be set up so that 
you will be allowed to attempt each problem three times, 
and you will get immediate feedback after each attempt.

\medskip

\noindent
{\bf Quizzes:}
There will be weekly quizzes on Thursdays, except
the Thursdays following exam Wednesdays.
The two lowest quiz scores will not count
towards the final grade.


\medskip

\noindent
{\bf Exams:}
There will be three midterm exams (Sep 20, Oct 18, Nov 15) and a final exam.  Each of the midterm exams are held on Wednesday evenings (5:15-6:45 pm, location TBA).  The final exam is on Tuesday, Dec 19, 4:30-7:00 pm (location TBA).  Calculators will not be allowed on the quizzes and the exams.  The final exam will be cumulative.  Make-up exams will be given only for unavoidable and documented absences.  Conflicts with work schedules, other classes, travel plans, etc.\ will not be considered unavoidable.

\medskip
%\vfill\eject

\noindent
{\bf Grading:}
You will be graded on your written work, which will be
judged on the basis of {\it correctness}, 
{\it completeness}, and {\it legibility}.
Your final grade will be determined by the scores of
your homework, quizzes, mid-term exams, and final exam.
To combine these items the following weights will be used:

\begin{tabular}[c]{lllllll}
{\bf Homework:} & 20\% &&&& {\bf Quizzes:} & 10\%\\
{\bf Midterm exams:} & 15\% each &&&& {\bf Final exam:} & 25\%\\
\end{tabular}

\noindent
There will be no extra credit assignments.

\medskip

\noindent
{\bf Getting Help:}
Don't wait until it is too late if you need help. {\it Ask questions!}  You can get help from me or your recitation TA during our office hours. If you can't see me during office hours,  then make an appointment 
with me to see me at a different time. You may also seek assistance at the Calculus Help Lab, which
will run from 4:00-6:00 p.m. Monday through Thursday (beginning September 5), in Math 170. 
The Help Lab will {\it not} run on exam Wednesdays or the following Thursdays. 
You may request help from any of the calculus tutors present at the lab.

\medskip

\noindent
{\bf Important Dates:}

%rlccrl
\begin{tabular}{llllll}
Sep 4 (Mon): & Labor Day &&&
   Oct 18 (Wed): & Second Midterm\\
Sep 13 (Wed): & 1st Drop Deadline &&& 
   Nov 20--22 (Mon--Wed): & Fall Break\\
Sep 20 (Wed): & 1st Midterm &&&
   Nov 23--24 (Thu--Fri): & Thanksgiving\\
Oct 11 (Wed): & 2nd Drop Deadline &&&
   Dec 15 (Fri): & Last day of classes\\
Nov 15 (Wed): & 3rd Midterm &&&
   Dec 19 (Tue): & Final Exam
\end{tabular}

\medskip

\noindent
{\bf Further Information About the Course:}
Information concerning this course will be posted on the course webpage.  Items announced in class have priority over anything written in this syllabus.

\medskip

\noindent
{\bf Policies:}
\begin{itemize}
\item
{\bf Limits of Collaboration:}
I recommend that you first attempt to solve the homework problems alone, 
using your textbook, your notes, and the resources of WileyPLUS.
If you need more help, you may consult me, your
recitation TA, a tutor in the Help Lab, or other persons. 
However, I require that you write your homework solutions unaided.  No collaboration of any type is permitted on quizzes or exams. 

\medskip
\item
By clicking on `Policies' on the course web page
you will find details about the following campus policies:
\begin{itemize}
\item 
{\bf Student Honor Code},
\item
{\bf Classroom Behavior}, 
\item
{\bf Accommodating Students with Disabilities}, and 
\item 
{\bf Observance of Religious Holidays}. 
\end{itemize}

\end{itemize}
If you need any special accommodation due to
medical disability or observance of a religious
holiday, please inform me as soon as possible
(preferably within the first two weeks of class),
and provide documentation. 

\medskip

\noindent {\bf Closing Remarks:}
When does the learning happen?  It might happen in class, but most likely it happens when you sit down to do your homework.  Most of you can follow what I do on the board, but the question is, can you do it on your own?  To learn best, you must struggle with mathematics on your own.  It is supposed to be difficult (if it is not difficult for you, then I will gladly find things to challenge you).  However, if you are struggling too much, then there are resources for you.  I am always happy to help you.  I want to help you.  If my office hours don't work for you, then we can probably find another time to meet.  You can also get help from each other.  Get a study buddy!  Help each other learn.  In addition, you can get help in Calculus Help Lab (MTWR, 4:00--6:00 p.m., Math 170). If you are having difficulty, then get some help.  It is your responsibility to be aware of how well you understand the material.  There are many resources available to you; use them!
\end{document}
