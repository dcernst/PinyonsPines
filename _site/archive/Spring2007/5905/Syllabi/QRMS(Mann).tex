% Course syllabus for QRMS section 005 Spring 2004
% LaTeX by Allen Mann

\documentclass[11pt]{article}
\usepackage{graphicx}
\usepackage{amssymb}

\textwidth = 6.5 in
\textheight = 9 in
\oddsidemargin = 0.0 in
\evensidemargin = 0.0 in
\topmargin = 0.0 in
\headheight = 0.0 in
\headsep = 0.0 in
\parskip = 0.2in
\parindent = 0.0in

\newtheorem{theorem}{Theorem}
\newtheorem{corollary}[theorem]{Corollary}
\newtheorem{definition}{Definition}

%Allen's Macro's
\newcommand{\AM}{\textsc{a.m.}}
\newcommand{\PM}{\textsc{p.m.}}
\newcommand{\book}[1]{\textit{#1}} 
\newcommand{\url}[1]{\texttt{#1}} 

\title{MATH 1012/QRMS 1010-005 \\ Quantitative Reasoning and Mathematical Skills}
\author{Course Syllabus}
\date{Spring 2004}
\begin{document}

\maketitle



\begin{tabbing}
\textit{Office Hours:} \quad   \= Allen.Mann@Colorado.EDU   \kill
\textit{Instructor:}		\> Allen Mann \\
\textit{Office Hours:}    	\> MWF 2:00--2:50 \PM,  T 1:00--1:50 \PM \\
\textit{Office:}			\> Mathematics 364 \\
\textit{Phone:}			\> (303) 492-3512 \\
\textit{E-mail:}			\> \url{Allen.Mann@Colorado.EDU} \\
\textit{Web:}			\> \url{http://ucsu.colorado.edu/\~\,$\!$almann/qrms}
\end{tabbing}

\begin{description}
\item[Time:] MWF 1:00--1:50 \PM

\item[Room:] MUEN E431

\item[Course Description:]
Promotes mathematical literacy among liberal arts students. Teaches basic mathematics, logic, and problem-solving skills in the context of higher level mathematics, science, technology, and/or society. This is not a traditional math class, but is designed to stimulate interest in and appreciation of mathematics and quantitative reasoning as valuable tools for comprehending the world in which we live. Meets MAPS requirement for mathematics. Approved for arts and sciences core curriculum: quantitative reasoning and mathematical skills.

\item[Textbook: ]
Jeffery O. Bennett and William L. Briggs. \book{Essentials of Using and Understanding Mathematics: A Quantitative Reasoning Approach}. Addison-Wesley, Boston, 2003.

\item[Course Requirements: ]
\begin{tabbing}
aaaaaaaaaaaaaaaaaaaaaaaaaaaaa\quad    \=Final Exam \quad  \=    \kill
\> Quizzes  \> 10\%    \\
\> Exams  \> 30\% \\
\> Paper   \> 30\% \\
\> Final Exam \> 30\% \\ 
\end{tabbing}

\vspace{-0.18 in}


\item[Homework \& Quizzes: ]
Homework exercises will not be graded. Instead, weekly quizzes will be given in class to assess students' understanding of the material. If you miss a quiz, you should come to office hours to make it up. You are encouraged to work together on homework exercises. All work on quizzes (and exams) must be your own without help from other students.

\item[Paper: ]
Each student will be required to write a paper on a mathematical topic. The paper should be limited to 1500 words. Students should turn in a hard copy of their paper April 12--16, either in class or to my mailbox located in Mathematics 260. Plagiarism will be reported to the University.

\item[Final Exam: ]
Wednesday, May 5, 1:30--4:00 \PM


\item[Grading: ]
All assignments will be graded using the following scale.
\begin{tabbing}
aaaaaaaaaaaaaaaaaaaaaaaaaaaaa\quad    \=Final Exam \quad  \=    \kill
\> 16--20  \> A    \\
\> 13--15  \> B    \\
\> 10--12  \> C    \\
\> 7--9  \> D    \\
\> 1--6  \> F    \\
\end{tabbing}

\item[Students with disabilities: ]
If you qualify for accommodations because of a disability, please submit to me a letter from Disability Services in a timely manner so that your needs may be addressed.  Disability Services determines accommodations based on documented disabilities. (303) 492-8671, Willard 322, \url{http://www.colorado.edu/sacs/disabilityservices}.

\item[Religious Obligations: ]
If you have a religious obligation that conflicts with one of the exams, please let me know at least two weeks in advance. Please refer to the University's policy on religious obligations at \url{http://www.colorado.edu/policies}.

\item[Classroom Behavior: ]
 See \url{http://www.colorado.edu/policies}.

\item[Honor Code: ] See \url{http://www.colorado.edu/academics/honorcode}.
\end{description}

\end{document}